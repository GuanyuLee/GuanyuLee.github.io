\documentclass{article}
\usepackage{amsmath}
\usepackage{amsfonts}
\usepackage{amssymb}
\usepackage{amsmath}
\usepackage{amsthm}
\usepackage{bm}
\usepackage{changepage}
\usepackage{esint}
\usepackage{framed}
\usepackage{geometry}
\usepackage{inputenc}
\usepackage{mathrsfs}
\usepackage{mathtools}
\usepackage{marvosym}
\usepackage{times}
\usepackage{xcolor}

\title{Math 4500 HW \#04 Solutions}
\author{Instructor: Birgit Speh\\ TA: Guanyu Li}
\date{}
\geometry{left=2cm,right=2cm,top=2.5cm,bottom=2.5cm}

\theoremstyle{definition}
\newtheorem{problem}{Problem}
\theoremstyle{plain}
\newtheorem*{remark}{Remark}

\begin{document}

\maketitle\par

\emph{This solution set is not error-free. Please email me (gl479\MVAt cornell.edu) if you spot any errors or typos!}

\begin{problem}[Exercise 3.4.1 (5 pts)]It is easy to test whether a matrix consists of blocks of the form
\begin{displaymath}
\begin{matrix}\alpha&-\beta\\ \bar{\beta}&\bar{\alpha}\end{matrix}.
\end{displaymath}
Nevertheless, it is sometimes convenient to describe the property of "being of the form $C(A)$" more algebraically. One way to do this is with the help of the special matrix
\begin{displaymath}
J=\begin{pmatrix}&1\\ -1&\end{pmatrix}.
\end{displaymath}
If $B:=\begin{pmatrix}\alpha&-\beta\\ \bar{\beta}&\bar{\alpha}\end{pmatrix}$, show that $JBJ^{-1}=\bar{B}$.
\end{problem}
\begin{adjustwidth}{0.7cm}{}
\color{blue}
\begin{proof}[Solution]
It is easy to see that $-JJ=\begin{pmatrix}&1\\ -1&\end{pmatrix}\begin{pmatrix}&1\\ -1&\end{pmatrix}=\begin{pmatrix}1&\\ &1\end{pmatrix}=I$ hence $J^{-1}=-J$. Thus
\begin{displaymath}
JBJ^{-1}=-\begin{pmatrix}&1\\ -1&\end{pmatrix}\begin{pmatrix}\alpha&-\beta\\ \bar{\beta}&\bar{\alpha}\end{pmatrix}\begin{pmatrix}&1\\ -1&\end{pmatrix}=-\begin{pmatrix}\bar{\beta}&\bar{\alpha}\\ -\alpha&\beta\end{pmatrix}\begin{pmatrix}&1\\ -1&\end{pmatrix}=\begin{pmatrix}\bar{\alpha}&-\bar{\beta}\\ \beta&\alpha\end{pmatrix}.
\end{displaymath}
\color{black}
\end{proof}
\end{adjustwidth}

\begin{problem}[Exercise 3.4.2 (5 pts)]Conversely, show that if $JBJ^{-1}=\bar{B}$ and $B:=\begin{pmatrix}c&d\\ e&f\end{pmatrix}$ then we have $\bar{c}=f$ and $\bar{d}=-e$, so $B$ has the form $\begin{pmatrix}\alpha&-\beta\\ \bar{\beta}&\bar{\alpha}\end{pmatrix}$.
\end{problem}
\begin{adjustwidth}{0.7cm}{}
\color{blue}
\begin{proof}[Solution]
Same as the calculation above,
\begin{displaymath}
JBJ^{-1}=-\begin{pmatrix}&1\\ -1&\end{pmatrix}\begin{pmatrix}c&d\\ e&f\end{pmatrix}\begin{pmatrix}&1\\ -1&\end{pmatrix}=\begin{pmatrix}-e&-f\\ c&d\end{pmatrix}\begin{pmatrix}&1\\ -1&\end{pmatrix}=\begin{pmatrix}f&-e\\ -d&c\end{pmatrix}.
\end{displaymath}
Since $JBJ^{-1}=\bar{B}$, we know
\begin{displaymath}
\begin{pmatrix}f&-e\\ -d&c\end{pmatrix}=\begin{pmatrix}\bar{c}&\bar{d}\\ \bar{e}&\bar{f}\end{pmatrix}.
\end{displaymath}
This is what we want.
\color{black}
\end{proof}
\end{adjustwidth}

\begin{problem}[Exercise 3.4.3 (13 pts)]Now suppose that $B_{2n}$ is any $2n\times2n$ complex matrix, and let
\begin{displaymath}
J_{2n}:=\begin{pmatrix}J&&\cdots&\\ &J&\cdots&\\ \vdots&\vdots&\ddots&\vdots\\ &&\cdots&J\end{pmatrix}.
\end{displaymath}
Use block multiplication, and the results of Exercises 3.4.1 and 3.4.2, to show that $B_{2n}$ has the form $C(A)$ if and only if $J_{2n}B_{2n}J_{2n}^{-1}=\bar{B}_{2n}$.
\end{problem}
\begin{adjustwidth}{0.7cm}{}
\color{blue}
\begin{proof}[Solution]Notice that
\begin{displaymath}
J_{2n}^{-1}:=\begin{pmatrix}J^{-1}&&\cdots&\\ &J^{-1}&\cdots&\\ \vdots&\vdots&\ddots&\vdots\\ &&\cdots&J^{-1}\end{pmatrix}=-J_{2n}.
\end{displaymath}
because the multiplication and inverse can be calculated by blocks. And also we have
\begin{align*}
J_{2n}B_{2n}J_{2n}^{-1}&=\begin{pmatrix}J&\cdots&\\ \vdots&\ddots&\vdots\\ &\cdots&J\end{pmatrix}\begin{pmatrix}Q_{1,1}&\cdots&Q_{1,n}\\ \vdots&\ddots&\vdots\\ Q_{n,1}&\cdots&Q_{n,n}\end{pmatrix}\begin{pmatrix}-J&\cdots&\\ \vdots&\ddots&\vdots\\ &\cdots&-J\end{pmatrix}\\
&=\begin{pmatrix}JQ_{1,1}&\cdots&JQ_{1,n}\\ \vdots&\ddots&\vdots\\ JQ_{n,1}&\cdots&JQ_{n,n}\end{pmatrix}\begin{pmatrix}-J&\cdots&\\ \vdots&\ddots&\vdots\\ &\cdots&-J\end{pmatrix}\\
&=\begin{pmatrix}JQ_{1,1}(-J)&\cdots&JQ_{1,n}(-J)\\ \vdots&\ddots&\vdots\\ JQ_{n,1}(-J)&\cdots&JQ_{n,n}(-J)\end{pmatrix}
\end{align*}
By problem 3.4.2, for each block $JQ_{i,j}J^{-1}=\overline{Q_{i,j}}$ iff so $Q_{i,j}$ has the form $\begin{pmatrix}\alpha&-\beta\\ \bar{\beta}&\bar{\alpha}\end{pmatrix}$, then so is $B_{2n}$.
\color{black}
\end{proof}
\end{adjustwidth}

\begin{problem}[Exercise 3.4.4 (7 pts)]By taking det of both sides of the equation in Exercise 3.4.3, show that $\det(B_{2n})$ is real.
\end{problem}
\begin{adjustwidth}{0.7cm}{}
\color{blue}
\begin{proof}[Solution]
Since we already know that $J_{2n}B_{2n}J_{2n}^{-1}=\bar{B}_{2n}$, by taking determinant we have $\det(J_{2n})\det(B_{2n})\det(J_{2n}^{-1})=\det(J_{2n}B_{2n}J_{2n}^{-1})=\det(\bar{B}_{2n})=\overline{\det(B_{2n})}$. Notice that $\det(J)=1$ and the determinant can be calculated by blocks,
\begin{displaymath}
\det(J_{2n})=\det(J)^{n}=1.
\end{displaymath}
Hence we have $\det(B_{2n})=\overline{\det(B_{2n})}$, which means $\det(B_{2n})$ is real.
\color{black}
\end{proof}
\end{adjustwidth}

\begin{problem}[Exercise 3.4.5 (5 pts)]Assuming now that $B_{2n}$ is in the complex form of $Sp(n)$, and hence is unitary, show that $\det(B_{2n})=\pm1$.
\end{problem}
\begin{adjustwidth}{0.7cm}{}
\color{blue}
\begin{proof}[Solution]Since $B_{2n}$ is in the complex form of $Sp(n)$, hence it is unitary and
\begin{displaymath}
B_{2n}\overline{B_{2n}^T}=I.
\end{displaymath}
Take determinant we have $\det(B_{2n})\overline{\det(B_{2n})}=1$, but since $\det(B_{2n})$ is real we know $\det(B_{2n})=\pm1$.
\color{black}
\end{proof}
\end{adjustwidth}

\begin{problem}[Exercise 3.3.2 (0 pts)]Show that vectors from an orthogonal basis of $\mathbb{C}^n$ if and only if their conjugates form an orthogonal basis, where the conjugate of a vector $(u_1,\cdots,u_n)$ is $(\bar{u_1},\cdots,\bar{u_n})$.
\end{problem}
\begin{adjustwidth}{0.7cm}{}
\color{blue}
\begin{proof}[Solution]By the reflexivity of conjugate, it suffices to prove that vectors from an orthogonal basis of $\mathbb{C}^n$ only if their conjugates form an orthogonal basis. Suppose $\{\bm{v}_1,\cdots,\bm{v}_n\}$ is an orthogonal basis of $\mathbb{C}^n$, then we know $\bm{v}_i^T\bar{\bm{v}}_j=\delta_{ij}$. Because $\delta_{ij}$ are always real, hence $\delta_{ij}=\bar{\delta_{ij}}=\overline{\bm{v}_i^T\bar{\bm{v}}_j}=\bar{\bm{v}_i}^T\bm{v}_j$, which means $\{\bar{\bm{v}}_1,\cdots,\bar{\bm{v}}_n\}$ is an orthogonal basis of $\mathbb{C}^n$.
\color{black}
\end{proof}
\end{adjustwidth}

\begin{problem}[Exercise 3.3.4 (0 pts)]Show that $A\bar{A}^T=I$ if and only if the column vectors of $A$ form an orthogonal basis.
\end{problem}
\begin{adjustwidth}{0.7cm}{}
\color{blue}
\begin{proof}[Solution]It is easy to see that $A\bar{A}^T=I$ iff $\bar{A}^TA=I$. Denote the column vectors of $A$ by $\bm{a}_1,\cdots,\bm{a}_n$, then $\bar{A}^TA=I$ implies $\bm{a}_i^T\bar{\bm{a}}_j=\delta_{ij}$, which means $\{\bm{a}_1,\cdots,\bm{a}_n\}$ forms an orthogonal basis.
\color{black}
\end{proof}
\end{adjustwidth}

\begin{problem}[Exercise 3.3.5 (0 pts)]Show that if $A$ preserves the Hermitian inner product, then the column vectors form an orthogonal basis.
\end{problem}
\begin{adjustwidth}{0.7cm}{}
\color{blue}
\begin{proof}[Solution]We denote the standard basis by $\{\bm{e}_1,\cdots,\bm{e}_n\}$, then since matrix $A$ preserve Hermitian product, $\delta_{ij}=\bm{e}_i^T\bar{\bm{e}}_j=(A\bm{e})_i^T\overline{A\bm{e}}_j=\bm{a}_i^T\bar{\bm{a}}_j$. Thus the column vectors form an orthogonal basis.
\color{black}
\end{proof}
\end{adjustwidth}

\end{document} 