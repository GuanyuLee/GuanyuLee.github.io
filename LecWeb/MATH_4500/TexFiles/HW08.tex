\documentclass{article}
\usepackage{amsmath}
\usepackage{amsfonts}
\usepackage{amssymb}
\usepackage{amsmath}
\usepackage{amsthm}
\usepackage{bm}
\usepackage{changepage}
\usepackage{esint}
\usepackage{framed}
\usepackage{geometry}
\usepackage{inputenc}
\usepackage{mathrsfs}
\usepackage{mathtools}
\usepackage{marvosym}
\usepackage{times}
\usepackage{xcolor}

\title{Math 4500 HW \#08 Solutions}
\author{Instructor: Birgit Speh\\ TA: Guanyu Li}
\date{}
\geometry{left=2cm,right=2cm,top=2.5cm,bottom=2.5cm}

\theoremstyle{definition}
\newtheorem{problem}{Problem}
\theoremstyle{plain}
\newtheorem*{remark}{Remark}

\begin{document}

\maketitle\par

\emph{This solution set is not error-free. Please email me (gl479\MVAt cornell.edu) if you spot any errors or typos!}

\begin{problem}[Exercise 5.6.4 (10 pts)]
When $X^2=-\det(X)I$, show that
\begin{displaymath}
e^X=\cos(\sqrt{\det(X)})I+\frac{\sin(\sqrt{\det(X)})}{\sqrt{\det(X)}}X.
\end{displaymath}
\end{problem}
\begin{adjustwidth}{0.7cm}{}
\color{blue}
\begin{proof}[Solution]
By definition
\begin{align*}
e^X&=\sum_{n=0}^{\infty}X^n\\
&=\sum_{n=0}^{\infty}\left(\frac{X^{2n}}{(2n)!}+\frac{X^{2n+1}}{(2n+1)!}\right)\\
&=\sum_{n=0}^{\infty}\left(\frac{X^{2n}}{(2n)!}+\frac{X^{2n}X}{(2n+1)!}\right)\\
&=\sum_{n=0}^{\infty}\left(\frac{(-\det(X))^n}{(2n)!}I+\frac{(-\det(X))^n}{(2n+1)!}X\right)\\
&=\sum_{n=0}^{\infty}\frac{(-\det(X))^n}{(2n)!}I+\sum_{n=0}^{\infty}\frac{(-\det(X))^n}{(2n+1)!}X\\
&=\cos(\sqrt{\det(X)})I+\frac{\sin(\sqrt{\det(X)})}{\sqrt{\det(X)}}X.
\end{align*}
\color{black}
\end{proof}
\end{adjustwidth}

\begin{problem}[Exercise 5.6.5 (10 pts)]
Using Exercise 5.6.4 and the fact that $\mathrm{Trace}(X)=0$, show that if
\begin{displaymath}
e^X=\begin{pmatrix}-1&1\\ &-1\end{pmatrix},
\end{displaymath}
then $\cos(\sqrt{\det(X)})=-1$, in which case $\sin(\sqrt{\det(X)})=0$, and there is a contradiction.
\end{problem}
\begin{adjustwidth}{0.7cm}{}
\color{blue}
\begin{proof}[Solution]
By Hamilton-Caylay, $\mathrm{Trace}(X)=0$ implies $X^2=-\det(X)I$. By the previous problem
\begin{displaymath}
\begin{pmatrix}-1&1\\ &-1\end{pmatrix}=e^X=\cos(\sqrt{\det(X)})I+\frac{\sin(\sqrt{\det(X)})}{\sqrt{\det(X)}}X.
\end{displaymath}
Take trace of both sides, then
\begin{displaymath}
-2=\mathrm{Trace}\left(\cos(\sqrt{\det(X)})I+\frac{\sin(\sqrt{\det(X)})}{\sqrt{\det(X)}}X\right).
\end{displaymath}
But $\mathrm{Trace}$ is linear so
\begin{displaymath}
-2=\cos(\sqrt{\det(X)})\mathrm{Trace}(I)+\frac{\sin(\sqrt{\det(X)})}{\sqrt{\det(X)}}\mathrm{Trace}(X)=2\cos(\sqrt{\det(X)}),
\end{displaymath}
hence $\cos(\sqrt{\det(X)})=-1$. Therefore $\sin(\sqrt{\det(X)})^2=1-\cos^2(\sqrt{\det(X)})=0$, thus $\sin(\sqrt{\det(X)})=0$. But this means
\begin{displaymath}
e^X=\cos(\sqrt{\det(X)})I+\frac{\sin(\sqrt{\det(X)})}{\sqrt{\det(X)}}X=\begin{pmatrix}-1&0\\ &-1\end{pmatrix},
\end{displaymath}
a contradiction.
\color{black}
\end{proof}
\end{adjustwidth}

\begin{problem}[Exercise 6.1.3 (5 pts)]
Show that $SU(n)$ is a normal subgroup of $U(n)$ by describing it as the kernel of some homomorphism.
\end{problem}
\begin{adjustwidth}{0.7cm}{}
\color{blue}
\begin{proof}[Solution]
We know that
\begin{displaymath}
\det:U(n)\to\mathbb{C}
\end{displaymath}
is a group homomorphism, because $\det(AB)=\det(A)\det(B)$. Hence the kernel of $\det$, which is $SU(n)$, is a normal subgroup.
\color{black}
\end{proof}
\end{adjustwidth}

\begin{problem}[Exercise 6.1.4 (5 pts)]
Show that $T_I(SU(n))$ is an ideal of $T_I(U(n))$ by checking that it has the required closure properties.
\end{problem}
\begin{adjustwidth}{0.7cm}{}
\color{blue}
\begin{proof}[Solution]
We know that
\begin{displaymath}
T_I(SU(n))=\{A\in M_n(\mathbb{C})\mid A+\bar{A}^T=0,\mathrm{Trace}(A)=0\}
\end{displaymath}
and
\begin{displaymath}
T_I(U(n))=\{A\in M_n(\mathbb{C})\mid A+\bar{A}^T=0\}.
\end{displaymath}
For any $A\in T_I(U(n))$ and $B\in T_I(SU(n))$,
\begin{align*}
\mathrm{Trace}~[A,B]=\mathrm{Trace}(AB-BA)=0
\end{align*}
since $\mathrm{Trace}~AB=\mathrm{Trace}~BA$. On the other hand
\begin{displaymath}
AB-BA+\overline{AB-BA}^T=AB-BA-\bar{A}^T\bar{B}^T-\bar{B}^T\bar{A}^T=0.
\end{displaymath}
Hence $T_I(SU(n))$ is an ideal of $T_I(U(n))$.
\color{black}
\end{proof}
\end{adjustwidth}

\begin{problem}[Exercise 6.3.4 (10 pts)]
Find a 1-dimensional ideal $J$ of $\mathfrak{u}(n)$, and show that $J$ is the tangent space of $Z(U(n))$.
\end{problem}
\begin{adjustwidth}{0.7cm}{}
\color{blue}
\begin{proof}[Solution]
Put $J:=\{\theta iI\mid\theta\in\mathbb{R}\}$. We have already know that $Z(U(n))=\{e^{i\theta}I\mid \theta\in\mathbb{R}\}$. Since $\alpha(t):=e^{i\theta t}I$ is a path in $Z(U(n))$ and $\alpha'(0)=\theta iI$, we know that $J\subseteq T_I(Z(U(n)))$. Conversely, for any path $\alpha(t)$ s.t. $\alpha(0)=I\in Z(U(n))$, it has the form $\alpha(t)=e^{i\theta(t)}I$ where $\theta(t)$ is a continuous map from $\mathbb{R}$ to $\mathbb{R}$ s.t. $\theta(0)=0$. Thus $\alpha'(0)=\theta'(0)iI\in J$. This proves that $J$ is an ideal. Apparently $J$ is of dimension 1, and it is an ideal since it is the tangent space of some normal subgroup.
\color{black}
\end{proof}
\end{adjustwidth}

\begin{problem}[Exercise 6.3.5 (5 pts)]
Also show that the $Z(U(n))$ is the image, under the exponential map, of the ideal $J$ in Exercise 6.3.4.
\end{problem}
\begin{adjustwidth}{0.7cm}{}
\color{blue}
\begin{proof}[Solution]
For any $e^{i\theta}I\in Z(U(n))$, we have $\theta iI\in J$ s.t. $e^{\theta iI}=e^{i\theta}I\in Z(U(n))$.
\color{black}
\end{proof}
\end{adjustwidth}

\begin{problem}[Exercise 5.5.1 (0 pts)]
\end{problem}
\begin{adjustwidth}{0.7cm}{}
\color{blue}
\begin{proof}[Solution]
\color{black}
\end{proof}
\end{adjustwidth}

\begin{problem}[Exercise 5.5.2 (0 pts)]
\end{problem}
\begin{adjustwidth}{0.7cm}{}
\color{blue}
\begin{proof}[Solution]
\color{black}
\end{proof}
\end{adjustwidth}

\begin{problem}[Exercise 5.5.4 (0 pts)]
\end{problem}
\begin{adjustwidth}{0.7cm}{}
\color{blue}
\begin{proof}[Solution]
\color{black}
\end{proof}
\end{adjustwidth}

\begin{problem}[Exercise 6.4.1 (0 pts)]
\end{problem}
\begin{adjustwidth}{0.7cm}{}
\color{blue}
\begin{proof}[Solution]
\color{black}
\end{proof}
\end{adjustwidth}

\begin{problem}[Exercise 6.4.2 (0 pts)]
\end{problem}
\begin{adjustwidth}{0.7cm}{}
\color{blue}
\begin{proof}[Solution]
\color{black}
\end{proof}
\end{adjustwidth}

\end{document} 