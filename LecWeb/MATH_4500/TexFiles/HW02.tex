\documentclass{article}
\usepackage{amsmath}
\usepackage{amsfonts}
\usepackage{amssymb}
\usepackage{amsmath}
\usepackage{mathrsfs}
\usepackage{bm}
\usepackage{changepage}
\usepackage{framed}
\usepackage{geometry}
\usepackage{inputenc}
\usepackage{amsthm}
\usepackage{mathtools}
\usepackage{esint}
\usepackage{marvosym}
\usepackage{xcolor}

\title{Math 4500 HW \#02 Solutions}
\author{Instructor: Birgit Speh\\ TA: Guanyu Li}
\date{}
\geometry{left=2cm,right=2cm,top=2.5cm,bottom=2.5cm}

\theoremstyle{definition}
\newtheorem{problem}{Problem}
\theoremstyle{plain}
\newtheorem*{remark}{Remark}

\begin{document}

\maketitle\par

\emph{This solution set is not error-free. Please email me (gl479\MVAt cornell.edu) if you spot any errors or typos!}

\begin{problem}[Exercise 2.1.4 (12 pts)]Show that there exists exactly one $n$-element subgroup of $SO(2)$, for each natural number $n$, and list its members.
\end{problem}
\begin{adjustwidth}{0.7cm}{}
\color{blue}
\begin{proof}[Solution]Existence is clear, since $K=\left\{\begin{pmatrix}\cos\frac{2\pi}{n}&\sin\frac{2\pi}{n}\\ -\sin\frac{2\pi}{n}&\cos\frac{2\pi}{n}\end{pmatrix},\cdots,\begin{pmatrix}\cos\frac{2\pi n}{n}&\sin\frac{2\pi n}{n}\\ -\sin\frac{2\pi n}{n}&\cos\frac{2\pi n}{n}\end{pmatrix}\right\}=\{e^\frac{2\pi i}{n},\cdots,e^\frac{2\pi in}{n}\}$ is a subgroup of $SO(2)$ consisting of exactly $n$ element$.^{[3]}$\par
To see the uniqueness, we first prove that for any $g$ in a finite group $G$, $g^{|G|}=1$. Consider the subgroup $H=\langle g\rangle$ generated by a single element $g$, if $g_1H$ and $g_2H$ are two cosets of $H$ whose intersection is not empty, then there is an $x\in G$ s.t. $x=g_1h_1=g_2h_2$ for some $h_1,h_2\in H$, implying $g_1=g_2h_2h_1^{-1}$. Thus for any $g_1h\in g_1H$, $g_1h=g_2h_2h_1^{-1}h\in g_2H$. This tells us $g_1H\subseteq g_2H$. Similarly, $g_2H\subseteq g_1H$. Hence $G$ can be partitioned as disjoint union of cosets of $H$. Hence $|H|\mid|G|$. Therefore $g^{|G|}=g^{k|\langle g\rangle|}=(g^{|\langle g\rangle|})^k$ for some integer $k$. It then suffices to prove $g^{|\langle g\rangle|}=1$. Suppose $|\langle g\rangle|=t$ then $\langle g\rangle=\{g^0,\cdots,g^{t-1}\}$ where the elements are all distinct. If $g^t\neq1$, one knows $g^t=g^i$ for some $1\leq i\leq t-1$. Thus $g^{t-i}=1$, which leads a contradiction since $g^0,\cdots,g^{t-1}$ are all distinct$.^{[6]}$\par
Thus, if $H$ is a subgroup of $SO(2)$, then by previous conclusion, every element $z\in H$ satisfies $z^n=1$. Therefore $z=e^\frac{2\pi ik}{n}$ for some integer $k$, which means $H$ is a subset of $K$. But $|H|=|K|=n$, implying $H=K$ as sets. Both multiplication of the two group comes from $SO(2)$, hence $H=K$ as subgroups$.^{[3]}$
\color{black}
\end{proof}
\end{adjustwidth}

\begin{problem}[Exercise 2.2.4 (5 pts)]Show that $\{\pm1\}$ is a normal subgroup of $S^3$.
\end{problem}
\begin{adjustwidth}{0.7cm}{}
\color{blue}
\begin{proof}[Solution]It suffices to prove for any $u\in S^3$, $u^{-1}(-1)u\in\{\pm1\}$. But
\begin{displaymath}
u^{-1}(-1)u=(-1)u^{-1}u=-1,
\end{displaymath}
hence $\{\pm1\}$ is a normal subgroup of $S^3$.
\color{black}
\end{proof}
\end{adjustwidth}

\begin{problem}[Exercise 2.2.5 (5 pts)]Show that $S^1$ is not a normal subgroup of $S^3$.
\end{problem}
\begin{adjustwidth}{0.7cm}{}
\color{blue}
\begin{proof}[Solution]It suffices to find an element $z$ in $S^1$, s.t. $qzq^{-1}\notin S^1$ for some $q\in S^3$. Let $z=i\in S^1$, and let $q=\frac{1+j}{\sqrt{2}}$. Then $q^{-1}=\frac{1-j}{\sqrt{2}}$. Hence
\begin{displaymath}
qzq^{-1}=\frac{1+j}{\sqrt{2}}i\frac{1-j}{\sqrt{2}}=\frac{i-k}{\sqrt{2}}\frac{1-j}{\sqrt{2}}=\frac{i-k-k-i}{2}=-k\notin S^1.
\end{displaymath}
\color{black}
\end{proof}
\end{adjustwidth}

\begin{problem}[Exercise 2.4.4 (10 pts)]Show that reflection in the hyperplane orthogonal to a coordinate axis has determinant -1, and generalize this result to any reflection.
\end{problem}
\begin{adjustwidth}{0.7cm}{}
\color{blue}
\begin{proof}[Solution]W.l.o.g., we first assume that the reflection is about the hyperplane $x_1x_2\cdots Ox_{n-1}$, and assume we have the standard basis $\{e_i\}_{i=1,\cdots,n}$. Denote the reflection as $r_n$, then the reflection about the hyperplane has the matrix
\begin{displaymath}
R=\begin{pmatrix}
1&&&&\\
&1&&&\\
&&\ddots&&\\
&&&1&\\
&&&&-1
\end{pmatrix}.
\end{displaymath}
Since the determinant is independent of the choice of orthonormal basis, we know $\det(r_n)=-1.^{[5]}$\par
Consider reflecting about an arbitrary plane $W$ in the space, suppose it passes through the origin. Then we can find an orthonormal basis of this plane $W$, say $\{\eta_1,\cdots,\eta_{n-1}\}$. Suppose $\eta_n$ is a unit vector perpendicular to the space $W$ and suppose $\{\eta_1,\cdots,\eta_n\}$ form an orthonormal basis for the space. We assume that
\begin{displaymath}
\eta_j=\sum_{i=1}^{n}a_{i,j}e_i,
\end{displaymath}
i.e.
\begin{displaymath}
(\eta_1,\cdots,\eta_n)=(e_1,\cdots,e_n)A.
\end{displaymath}
Thus the matrix of reflection $r_n$ under the basis $\{\eta_1,\cdots,\eta_n\}$ is $A^TRA$, hence the determinant is
\begin{displaymath}
\det(A^TRA)=\det(A^T)\det(R)\det(A)=-1
\end{displaymath}
since $\det(A^T)=\det(A)=1.^{[5]}$
\color{black}
\end{proof}
\end{adjustwidth}

\begin{problem}[Exercise 2.6.3 (15 pts)]Observe that the rotations in Exercise 2.6.1 form an $S^1$, as do the rotations in Exercise 2.6.2, and deduce that $SO(4)$ contains a subgroup isomorphic to $T^2$.
\end{problem}
\begin{adjustwidth}{0.7cm}{}
\color{blue}
\begin{proof}[Solution]We construct the map to show they are isomorphic. Let
\begin{align*}
\varpi:S^1\times S^1&\to SO(4)\\
\left(\begin{pmatrix}\cos\frac{2\pi\theta}{n}&\sin\frac{2\pi\theta}{n}\\ -\sin\frac{2\pi\theta}{n}&\cos\frac{2\pi\theta}{n}\end{pmatrix},\begin{pmatrix}\cos\frac{2\pi\phi}{n}&\sin\frac{2\pi\phi}{n}\\ -\sin\frac{2\pi\phi}{n}&\cos\frac{2\pi\phi}{n}\end{pmatrix}\right)&\mapsto\begin{pmatrix}\cos\frac{2\pi\theta}{n}&\sin\frac{2\pi\theta}{n}&&\\ -\sin\frac{2\pi\theta}{n}&\cos\frac{2\pi\theta}{n}&&\\ &&\cos\frac{2\pi\phi}{n}&\sin\frac{2\pi\phi}{n}\\ &&-\sin\frac{2\pi\phi}{n}&\cos\frac{2\pi\phi}{n}\end{pmatrix}.^{[5]}
\end{align*}
Suppose $\left(\begin{pmatrix}\cos\frac{2\pi\theta_1}{n}&\sin\frac{2\pi\theta_1}{n}\\ -\sin\frac{2\pi\theta_1}{n}&\cos\frac{2\pi\theta_1}{n}\end{pmatrix},\begin{pmatrix}\cos\frac{2\pi\phi_1}{n}&\sin\frac{2\pi\phi_1}{n}\\ -\sin\frac{2\pi\phi_1}{n}&\cos\frac{2\pi\phi_1}{n}\end{pmatrix}\right)$ and $\left(\begin{pmatrix}\cos\frac{2\pi\theta_2}{n}&\sin\frac{2\pi\theta_2}{n}\\ -\sin\frac{2\pi\theta_2}{n}&\cos\frac{2\pi\theta_2}{n}\end{pmatrix},\begin{pmatrix}\cos\frac{2\pi\phi_2}{n}&\sin\frac{2\pi\phi_2}{n}\\ -\sin\frac{2\pi\phi_2}{n}&\cos\frac{2\pi\phi_2}{n}\end{pmatrix}\right)$ are two elements in $S^1\times S^1$, then
\begin{align*}
&\varpi\left(\left(\begin{pmatrix}\cos\frac{2\pi\theta_1}{n}&\sin\frac{2\pi\theta_1}{n}\\ -\sin\frac{2\pi\theta_1}{n}&\cos\frac{2\pi\theta_1}{n}\end{pmatrix},\begin{pmatrix}\cos\frac{2\pi\phi_1}{n}&\sin\frac{2\pi\phi_1}{n}\\ -\sin\frac{2\pi\phi_1}{n}&\cos\frac{2\pi\phi_1}{n}\end{pmatrix}\right)\left(\begin{pmatrix}\cos\frac{2\pi\theta_2}{n}&\sin\frac{2\pi\theta_2}{n}\\ -\sin\frac{2\pi\theta_2}{n}&\cos\frac{2\pi\theta_2}{n}\end{pmatrix},\begin{pmatrix}\cos\frac{2\pi\phi_2}{n}&\sin\frac{2\pi\phi_2}{n}\\ -\sin\frac{2\pi\phi_2}{n}&\cos\frac{2\pi\phi_2}{n}\end{pmatrix}\right)\right)\\
=&\varpi\left(\left(\begin{pmatrix}\cos\frac{2\pi\theta_1}{n}&\sin\frac{2\pi\theta_1}{n}\\ -\sin\frac{2\pi\theta_1}{n}&\cos\frac{2\pi\theta_1}{n}\end{pmatrix}\begin{pmatrix}\cos\frac{2\pi\theta_2}{n}&\sin\frac{2\pi\theta_2}{n}\\ -\sin\frac{2\pi\theta_2}{n}&\cos\frac{2\pi\theta_2}{n}\end{pmatrix},\begin{pmatrix}\cos\frac{2\pi\phi_1}{n}&\sin\frac{2\pi\phi_1}{n}\\ -\sin\frac{2\pi\phi_1}{n}&\cos\frac{2\pi\phi_1}{n}\end{pmatrix}\begin{pmatrix}\cos\frac{2\pi\phi_2}{n}&\sin\frac{2\pi\phi_2}{n}\\ -\sin\frac{2\pi\phi_2}{n}&\cos\frac{2\pi\phi_2}{n}\end{pmatrix}\right)\right)\\
=&\varpi\left(\left(\begin{pmatrix}\cos\frac{2\pi(\theta_1+\theta_2)}{n}&\sin\frac{2\pi(\theta_1+\theta_2)}{n}\\ -\sin\frac{2\pi(\theta_1+\theta_2)}{n}&\cos\frac{2\pi(\theta_1+\theta_2)}{n}\end{pmatrix},\begin{pmatrix}\cos\frac{2\pi(\phi_1+\phi_2)}{n}&\sin\frac{2\pi(\phi_1+\phi_2)}{n}\\ -\sin\frac{2\pi(\phi_1+\phi_2)}{n}&\cos\frac{2\pi(\phi_1+\phi_2)}{n}\end{pmatrix}\right)\right)\\
=&\begin{pmatrix}\cos\frac{2\pi(\theta_1+\theta_2)}{n}&\sin\frac{2\pi(\theta_1+\theta_2)}{n}&&\\ -\sin\frac{2\pi(\theta_1+\theta_2)}{n}&\cos\frac{2\pi(\theta_1+\theta_2)}{n}&&\\ &&\cos\frac{2\pi(\phi_1+\phi_2)}{n}&\sin\frac{2\pi(\phi_1+\phi_2)}{n}\\ &&-\sin\frac{2\pi(\phi_1+\phi_2)}{n}&\cos\frac{2\pi(\phi_1+\phi_2)}{n}\end{pmatrix},
\end{align*}
and
\begin{align*}
&\varpi\left(\begin{pmatrix}\cos\frac{2\pi\theta_1}{n}&\sin\frac{2\pi\theta_1}{n}\\ -\sin\frac{2\pi\theta_1}{n}&\cos\frac{2\pi\theta_1}{n}\end{pmatrix},\begin{pmatrix}\cos\frac{2\pi\phi_1}{n}&\sin\frac{2\pi\phi_1}{n}\\ -\sin\frac{2\pi\phi_1}{n}&\cos\frac{2\pi\phi_1}{n}\end{pmatrix}\right)\varpi\left(\begin{pmatrix}\cos\frac{2\pi\theta_2}{n}&\sin\frac{2\pi\theta_2}{n}\\ -\sin\frac{2\pi\theta_2}{n}&\cos\frac{2\pi\theta_2}{n}\end{pmatrix},\begin{pmatrix}\cos\frac{2\pi\phi_2}{n}&\sin\frac{2\pi\phi_2}{n}\\ -\sin\frac{2\pi\phi_2}{n}&\cos\frac{2\pi\phi_2}{n}\end{pmatrix}\right)\\
=&\begin{pmatrix}\cos\frac{2\pi\theta_1}{n}&\sin\frac{2\pi\theta_1}{n}&&\\ -\sin\frac{2\pi\theta_1}{n}&\cos\frac{2\pi\theta_1}{n}&&\\ &&\cos\frac{2\pi\phi_1}{n}&\sin\frac{2\pi\phi_1}{n}\\ &&-\sin\frac{2\pi\phi_1}{n}&\cos\frac{2\pi\phi_1}{n}\end{pmatrix}\begin{pmatrix}\cos\frac{2\pi\theta_2}{n}&\sin\frac{2\pi\theta_2}{n}&&\\ -\sin\frac{2\pi\theta_2}{n}&\cos\frac{2\pi\theta_2}{n}&&\\ &&\cos\frac{2\pi\phi_2}{n}&\sin\frac{2\pi\phi_2}{n}\\ &&-\sin\frac{2\pi\phi_2}{n}&\cos\frac{2\pi\phi_2}{n}\end{pmatrix}\\
=&\begin{pmatrix}\cos\frac{2\pi(\theta_1+\theta_2)}{n}&\sin\frac{2\pi(\theta_1+\theta_2)}{n}&&\\ -\sin\frac{2\pi(\theta_1+\theta_2)}{n}&\cos\frac{2\pi(\theta_1+\theta_2)}{n}&&\\ &&\cos\frac{2\pi(\phi_1+\phi_2)}{n}&\sin\frac{2\pi(\phi_1+\phi_2)}{n}\\ &&-\sin\frac{2\pi(\phi_1+\phi_2)}{n}&\cos\frac{2\pi(\phi_1+\phi_2)}{n}\end{pmatrix}.
\end{align*}
Hence $\varpi$ is a group homomorphism$.^{[5]}$\par
Suppose $T^2$ be the image of $\varpi$, it suffices to prove $\varpi$ is injective then $T^2$ is the subgroup contained in $SO(4)$. And it suffices to prove the kernel is trivial. Clearly, if
\begin{displaymath}
\varpi\left(\begin{pmatrix}\cos\frac{2\pi\theta}{n}&\sin\frac{2\pi\theta}{n}\\ -\sin\frac{2\pi\theta}{n}&\cos\frac{2\pi\theta}{n}\end{pmatrix},\begin{pmatrix}\cos\frac{2\pi\phi}{n}&\sin\frac{2\pi\phi}{n}\\ -\sin\frac{2\pi\phi}{n}&\cos\frac{2\pi\phi}{n}\end{pmatrix}\right)=\begin{pmatrix}1&&&\\ &1&&\\ &&1&\\ &&&1\end{pmatrix},
\end{displaymath}
then we must have $\theta,\phi=k$ for some integer $k$. Thus $\left(\begin{pmatrix}\cos\frac{2\pi\theta}{n}&\sin\frac{2\pi\theta}{n}\\ -\sin\frac{2\pi\theta}{n}&\cos\frac{2\pi\theta}{n}\end{pmatrix},\begin{pmatrix}\cos\frac{2\pi\phi}{n}&\sin\frac{2\pi\phi}{n}\\ -\sin\frac{2\pi\phi}{n}&\cos\frac{2\pi\phi}{n}\end{pmatrix}\right)$ is the identity in $S^1\times S^1.^{[5]}$
\color{black}
\end{proof}
\end{adjustwidth}

\begin{problem}[Exercise 2.6.5 (8 pts)]Explain why $S^3=SU(2)$ is not the same as $S^1\times S^1\times S^1$.
\end{problem}
\begin{adjustwidth}{0.7cm}{}
\color{blue}
\begin{proof}[Solution 1]First we project $S^3\subseteq\mathbb{R}^4$ onto the hyperplane perpendicular to $z$-axis, then the image is $S^2\subseteq\mathbb{R}^3$. On the other hand, $S^1\times S^1$ is the torus $T^2$, we embed it into $\mathbb{R}^3$ via
\begin{displaymath}
f(x,y,z)=z^2+(\sqrt{x^2+y^2}-\frac{1}{2})^2=1.
\end{displaymath}
Project it onto $xOy$ plane, then we have the region
\begin{displaymath}
D:=\{(x,y)\mid\frac{1}{2}\leq x^2+y^2\leq\frac{3}{2}\}.
\end{displaymath}
Therefore, $D\times S^1$ should be the projection of $S^1\times S^1\times S^1\hookrightarrow\mathbb{R}^4$ onto the hyperplane perpendicular to $z$-axis at the origin. But we can see there is a "hole", which cannot be the case of $S^2\subseteq\mathbb{R}^3$. Hence $S^3=SU(2)$ and $S^1\times S^1\times S^1$ are different geometrical objects.
\color{black}
\end{proof}
\end{adjustwidth}
\begin{remark}
The real intuition of proving these two space are not the same thing comes from topology. Usually it is extremely hard to say there is no homeomorphism (loosely some prerequisite to be an isomorphism of Lie groups) between two spaces technically. However, some really smart guys said, well, we could do it in another way. We could construct some \textbf{invariant} so that the same spaces, or equivalent spaces should have the same invariant, and if we calculated that two spaces have two different invariant, then they must be different. Here in this situation, a natural way to say is $\pi_1(S^3)=1$ while $\pi_1(S^1\times S^1\times S^1)=\mathbb{Z}^3$. Also, the intuitive number of holes is also an invariant, which should be defined as half of the dimension of $H^1$ over $\mathbb{R}$.
\end{remark}
\begin{adjustwidth}{0.7cm}{}
\color{blue}
\begin{proof}[Solution 2]We know $i,j\in S^3$ and $ij\neq ji$.\par
But for any $a,b\in S^1$, $ab=ba$, we know
\begin{displaymath}
(a_1,a_2,a_3)(b_1,b_2,b_3)=(a_1b_1,a_2b_2,a_3b_3)=(b_1a_1,b_2a_2,b_3a_3)=(b_1,b_2,b_3)(a_1,a_2,a_3)
\end{displaymath}
for any $(a_1,a_2,a_3),(b_1,b_2,b_3)\in S^1\times S^1\times S^1$. Hence they are different groups.
\color{black}
\end{proof}
\end{adjustwidth}

\begin{problem}[Exercise 2.2.3 (0 pts)]Show that the map $z\mapsto z^2$ is a well-defined map from $G$ to $S^1$, and that the map is an isomorphism.
\end{problem}
\begin{adjustwidth}{0.7cm}{}
\color{blue}
\begin{proof}[Solution]First we need to verify that the map is well-defined. Suppose we have two representatives $+z_\alpha$ and $-z_\alpha$, then it is clear $(+z_\alpha)^2=(-z_\alpha)^2$, hence it is well-defined.\par
Suppose $\{\pm z_\alpha\},\{\pm z_\beta\}$ are two elements in $G$, then
\begin{displaymath}
(\{\pm z_\alpha\}\{\pm z_\beta\})^2=(\{\pm z_\alpha z_\beta\})^2=z_\alpha^2z_\beta^2=(\{\pm z_\alpha\})^2(\{\pm z_\beta\})^2.
\end{displaymath}
Hence it is a homomorphism.
\end{proof}
\color{black}
\end{adjustwidth}

\begin{problem}[Exercise 2.4.2 (0 pts)]Using the fact that $u+v$ is the midpoint of the line joining $2u$ and $2v$, and Exercise 2.4.2, show that $f(u+v)=f(u)+f(v)$.
\end{problem}
\begin{adjustwidth}{0.7cm}{}
\color{blue}
\begin{proof}[Solution]Suppose $u,v$ are two vectors, then they are the midpoint of $2u$ and $2v$ respectively. Since $f$ preserves straight lines and midpoints of line segments, the midpoint of $2u$ and $2v$ is the midpoint of $2u$ joining the midpoint of $2v$, which means $f(u+v)=f(u)+f(v)$.
\color{black}
\end{proof}
\end{adjustwidth}

\begin{problem}[Exercise 2.4.3 (0 pts)]Also prove that $f(ru)=rf(u)$ for any real number $r$.
\end{problem}
\begin{adjustwidth}{0.7cm}{}
\color{blue}
\begin{proof}[Solution]First by definition, $f(ku)=kf(u)$ for all integer $k$. Suppose $r\in\mathbb{Q}$ s.t. $r=\frac{a}{b}$, then $f(bru)=brf(u)$ and $f(bru)=bf(ru)$. Hence $f(ru)=rf(u)$ for any rational number $r$. Finally, suppose $t$ is an arbitrary real number then we have a sequence of rational numbers $\{r_n\}$ s.t. $r_n\to t$. Therefore $f(tu)=f(\lim_{n\to\infty}r_nu)=\lim_{n\to\infty}f(r_nu)=\lim_{n\to\infty}r_nf(u)=tf(u)$.
\color{black}
\end{proof}
\end{adjustwidth}

\end{document} 