\documentclass{article}
\usepackage{amsmath}
\usepackage{amsfonts}
\usepackage{amssymb}
\usepackage{amsmath}
\usepackage{amsthm}
\usepackage{bm}
\usepackage{changepage}
\usepackage{esint}
\usepackage{framed}
\usepackage{geometry}
\usepackage{inputenc}
\usepackage{mathrsfs}
\usepackage{mathtools}
\usepackage{marvosym}
\usepackage{times}
\usepackage{xcolor}

\title{Math 4500 HW \#13 Solutions}
\author{Instructor: Birgit Speh\\ TA: Guanyu Li}
\date{}
\geometry{left=2cm,right=2cm,top=2.5cm,bottom=2.5cm}

\theoremstyle{definition}
\newtheorem{problem}{Problem}
\theoremstyle{plain}
\newtheorem*{remark}{Remark}

\begin{document}

\maketitle\par

\emph{This solution set is not error-free. Please email me (gl479\MVAt cornell.edu) if you spot any errors or typos!}

\begin{problem}[Exercise 9.3.1 (10 pts)]
Using the fact that $\Phi$ is a group homomorphism, show that we also have $(\Phi\circ C)'(0)=(\Phi\circ A)'(0)+(\Phi\circ B)'(0)$ where $C(t)=A(t)B(t)$.
\end{problem}
\begin{adjustwidth}{0.7cm}{}
\color{blue}
\begin{proof}[Solution]
Since $\Phi$ is a homomorphism, we have
\begin{align*}
(\Phi\circ C(t))'(0)&=(\Phi\circ A(t)B(t))'(0)\\
&=((\Phi\circ A(t))(\Phi\circ B(t)))'(0)\\
&=(\Phi\circ A(t))'(0)(\Phi\circ B(t))(0)+(\Phi\circ A(t))(0)(\Phi\circ B(t))'(0)\\
&=(\Phi\circ A(t))'(0)+(\Phi\circ B(t))'(0).
\end{align*}
\color{black}
\end{proof}
\end{adjustwidth}

\begin{problem}[Exercise 9.3.2 (5 pts)]
Deduce from Exercise 9.3.1 that $\varphi(A'(0)+B'(0))=\varphi(A'(0))+\varphi(B'(0))$.
\end{problem}
\begin{adjustwidth}{0.7cm}{}
\color{blue}
\begin{proof}[Solution]
Let $C(t)=A(t)B(t)$, then $C'(t)=A'(t)+B'(t)$. By previous problem, we are done.
\color{black}
\end{proof}
\end{adjustwidth}

\begin{problem}[Exercise 9.3.3 (10 pts)]
Let $D(t)=A(rt)$ for some real number $r$. Show that $D'(0)=rA'(0)$ and $(\Phi\circ D)'(0)=r(\Phi\circ A)'(0)$.
\end{problem}
\begin{adjustwidth}{0.7cm}{}
\color{blue}
\begin{proof}[Solution]
By the chain rule,
\begin{displaymath}
D'(t)=(A(rt))'=rA'(rt),
\end{displaymath}
hence $D'(0)=rA'(0)$. On the other hand, let $\Psi:=\Phi\circ A$, then
\begin{align*}
(\Phi\circ D(t))'(0)&=(\Psi\circ rt))'(0)\\
&=r(\Psi\circ t))'(0)=rD'(0).
\end{align*}
\color{black}
\end{proof}
\end{adjustwidth}

\begin{problem}[Exercise 9.3.4 (5 pts)]
Deduce from Exercise 9.3.2 and 9.3.3 that $\varphi$ is linear.
\end{problem}
\begin{adjustwidth}{0.7cm}{}
\color{blue}
\begin{proof}[Solution]
Directly from previous exercises.
\color{black}
\end{proof}
\end{adjustwidth}

\begin{problem}Classify all the 2-dimensional Lie algebras and find their groups.
\end{problem}
\begin{adjustwidth}{0.7cm}{}
\color{blue}
\begin{proof}[Solution]
$\mathbb{C}^2$ with trivial Lie bracket, and its group can be $\mathbb{C}^2$. The other one is $\mathbb{C}^2$ with Lie bracket defined in Page 89. Its group is $\mathrm{Aff}(1)$.
\color{black}
\end{proof}
\end{adjustwidth}

\begin{problem}[Exercise 8.7.3 (0 pts)]
\end{problem}
\begin{adjustwidth}{0.7cm}{}
\color{blue}
\begin{proof}[Solution]
\color{black}
\end{proof}
\end{adjustwidth}

\begin{problem}[Exercise 8.7.4 (0 pts)]
\end{problem}
\begin{adjustwidth}{0.7cm}{}
\color{blue}
\begin{proof}[Solution]
\color{black}
\end{proof}
\end{adjustwidth}

\begin{problem}[Exercise 8.7.5 (0 pts)]
\end{problem}
\begin{adjustwidth}{0.7cm}{}
\color{blue}
\begin{proof}[Solution]
\color{black}
\end{proof}
\end{adjustwidth}

\end{document} 