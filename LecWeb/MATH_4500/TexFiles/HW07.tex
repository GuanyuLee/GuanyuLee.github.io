\documentclass{article}
\usepackage{amsmath}
\usepackage{amsfonts}
\usepackage{amssymb}
\usepackage{amsmath}
\usepackage{amsthm}
\usepackage{bm}
\usepackage{changepage}
\usepackage{esint}
\usepackage{framed}
\usepackage{geometry}
\usepackage{inputenc}
\usepackage{mathrsfs}
\usepackage{mathtools}
\usepackage{marvosym}
\usepackage{times}
\usepackage{xcolor}

\title{Math 4500 HW \#07 Solutions}
\author{Instructor: Birgit Speh\\ TA: Guanyu Li}
\date{}
\geometry{left=2cm,right=2cm,top=2.5cm,bottom=2.5cm}

\theoremstyle{definition}
\newtheorem{problem}{Problem}
\theoremstyle{plain}
\newtheorem*{remark}{Remark}

\begin{document}

\maketitle\par

\emph{This solution set is not error-free. Please email me (gl479\MVAt cornell.edu) if you spot any errors or typos!}

\begin{problem}[Exercise 4.3.5 (7 pts)]Use bilinearity, or otherwise, show that $U,V\in\mathbb{R}\bm{i}+\mathbb{R}\bm{j}+\mathbb{R}\bm{k}$ implies $[U,V]\in\mathbb{R}\bm{i}+\mathbb{R}\bm{j}+\mathbb{R}\bm{k}$.
\end{problem}
\begin{adjustwidth}{0.7cm}{}
\color{blue}
\begin{proof}[Solution]
Suppose $U=a\bm{i}+b\bm{j}+c\bm{k},V=x\bm{i}+y\bm{j}+z\bm{k}$. By definition of quaternion multiplication
\begin{align*}
[U,V]&=UV-VU=(a\bm{i}+b\bm{j}+c\bm{k})(x\bm{i}+y\bm{j}+z\bm{k})-(x\bm{i}+y\bm{j}+z\bm{k})(a\bm{i}+b\bm{j}+c\bm{k})\\
&=(-ax+ay\bm{k}-az\bm{j}-by+bz\bm{i}-bx\bm{k}-cz+cx\bm{j}-cy\bm{i})\\
&-(-ax-ay\bm{k}+az\bm{j}-by-bz\bm{i}+bx\bm{k}-cz-cx\bm{j}+cy\bm{i})\\
&=2(bz-cy)\bm{i}+2(cx-az)\bm{j}+2(ay-bx)\bm{k}\in\mathbb{R}\bm{i}+\mathbb{R}\bm{j}+\mathbb{R}\bm{k}.
\end{align*}
\color{black}
\end{proof}
\end{adjustwidth}

\begin{problem}[Exercise 4.4.1 (3 pts)]Prove the Jacobi identity by using the definition $[X,Y]=XY-YX$ to expand $[X,[Y,Z]]+[Z,[X,Y]]+[Y,[Z,X]]$.
\end{problem}
\begin{adjustwidth}{0.7cm}{}
\color{blue}
\begin{proof}[Solution]
This verification comes from the definition
\begin{align*}
[X,[Y,Z]]+[Z,[X,Y]]+[Y,[Z,X]]&=[X,YZ-ZY]+[Z,XY-YX]+[Y,ZX-XZ]\\
&=XYZ-XZY-YZX+ZYX\\
&+ZXY-ZYX-XYZ+YXZ\\
&+YZX-YXZ-ZXY+XZY\\
&=0.
\end{align*}
\color{black}
\end{proof}
\end{adjustwidth}

\begin{problem}[Exercise 5.2.8 (15 pts)]Deduce from Exercise 5.2.6 and 5.2.7 that each matrix in $SO(3)$ equals $e^X$ for some skew-symmetric $X$.
\end{problem}
\begin{adjustwidth}{0.7cm}{}
\color{blue}
\begin{proof}[Solution]
First we compute $e^B$ for $B=\begin{pmatrix}&-\theta&\\ \theta&&\\ &&\end{pmatrix}$. Denote $P=\begin{pmatrix}&-1&\\ 1&&\\ &&\end{pmatrix}$, then we know that
\begin{displaymath}
e^B=e^{\theta P}=\sum_{n=0}^{\infty}(\theta P)^n=\begin{pmatrix}\cos\theta&-\sin\theta&\\ \sin\theta&\cos\theta&\\ &&1\end{pmatrix}.
\end{displaymath}\par
Suppose $A$ is an orthogonal matrix, then
\begin{align*}
Ae^BA^T&=A\left(\sum_{n=0}^{\infty}B^n\right)A^T\\
&=\sum_{n=0}^{\infty}AB^nA^T\\
&=\sum_{n=0}^{\infty}(ABA^T)^n\\
&=e^{ABA^T}.
\end{align*}
We know that for any orthogonal matrix $A\in O(n)$, we have a decomposition
\begin{displaymath}
A=CHC^T
\end{displaymath}
where $C\in O(n)$, $H$ is a block-diagonal matrix having the form $\mathrm{diag}\{R_1,\cdots,R_m,1,\cdots,1\}$ and $R_i=\begin{pmatrix}\cos\theta_i&-\sin\theta_i\\ \sin\theta_i&\cos\theta_i\end{pmatrix}$ for some real $\theta_i$. Here in $\mathbb{R}^3$
\begin{displaymath}
A=C\begin{pmatrix}\cos\theta&-\sin\theta&\\ \sin\theta&\cos\theta&\\ &&1\end{pmatrix}C^T=Ce^BC^T=e^{CBC^T},
\end{displaymath}
where $C\in O(3)$ and thus $CBC^T\in SO(3)$.
\color{black}
\end{proof}
\end{adjustwidth}

\begin{problem}[Exercise 5.3.6 (13 pts)]
Show that the skew-Hermitian matrices in the tangent space of $SU(2)$ can be written in the form $b\bm{i}+c\bm{j}+d\bm{k}$ where $b,c,d\in\mathbb{R}$ and $\bm{i},\bm{j},\bm{k}$ are matrices with the same multiplication table as the quaternions $\bm{i},\bm{j},\bm{k}$.
\end{problem}
\begin{adjustwidth}{0.7cm}{}
\color{blue}
\begin{proof}[Solution]
Notice that all the matrices in the tangent space have the form
\begin{displaymath}
A=\begin{pmatrix}di&b+ci\\ -b+ci&-di\end{pmatrix},
\end{displaymath}
since $A+\bar{A}^T=0$ and $\mathrm{Trace}(A)=0$. First we compute the linear combination for specific $\bm{i}_0=\begin{pmatrix}&-1\\1&\end{pmatrix},\bm{j}_0=\begin{pmatrix}&-i\\-i&\end{pmatrix},\bm{k}_0=\begin{pmatrix}i&\\&-i\end{pmatrix}$. It is obviously that
\begin{align*}
A&=\begin{pmatrix}di&b+ci\\ -b-ci&-di\end{pmatrix}\\
&=-b\begin{pmatrix}&-1\\1&\end{pmatrix}-c\begin{pmatrix}&-i\\-i&\end{pmatrix}+d\begin{pmatrix}i&\\&-i\end{pmatrix}\\
&=-b\bm{i}_0-c\bm{j}_0+d\bm{k}_0.
\end{align*}\par
Then for arbitrary matrices with the same multiplication table as the quaternions $\bm{i},\bm{j},\bm{k}$, we have some matrix $C\in SO(3)$ s.t. $C[\bm{i},\bm{j},\bm{k}]=[\bm{i}_0,\bm{j}_0,\bm{k}_0]$, i.e. we have some change of basis s.t. the multiplication table of bases is preserved. Hence
\begin{align*}
A&=-b\bm{i}_0-c\bm{j}_0+d\bm{k}_0\\
&=-b'\bm{i}-c'\bm{j}+d'\bm{k}
\end{align*}
where $\begin{pmatrix}b'\\ c'\\ d'\end{pmatrix}=C\begin{pmatrix}b\\ c\\ d\end{pmatrix}$.
\color{black}
\end{proof}
\end{adjustwidth}

\begin{problem}[Exercise 5.3.7 (10 pts)]
Also find the tangent space of $Sp(1)$.
\end{problem}
\begin{adjustwidth}{0.7cm}{}
\color{blue}
\begin{proof}[Solution]
Suppose $q(t)$ be a smooth path of $Sp(1)$ originating at $I$, then
\begin{displaymath}
q(t)\overline{q(t)}=I.
\end{displaymath}
Take the derivative, then
\begin{displaymath}
q'(t)\overline{q(t)}+q(t)\overline{q'(t)}=0.
\end{displaymath}
Since $q(t)=I$, for $t=0$ the equation becomes
\begin{displaymath}
q'(0)+\overline{q'(0)}=0,
\end{displaymath}
hence the tangent vector should be a pure imaginary quaternion. Conversely, for any pure imaginary quaternion $p=b\bm{i}+c\bm{j}+d\bm{k}$, set
\begin{displaymath}
q(t):=e^{tp}\in Sp(1)
\end{displaymath}
where $t\in[-1,1]$, then apparently $q'(0)=p=b\bm{i}+c\bm{j}+d\bm{k}$.
\color{black}
\end{proof}
\end{adjustwidth}

\begin{problem}[Exercise 5.3.8 (7 pts)]
Prove that $\mathrm{Tr}(XY)=\mathrm{Tr}(YX)$.
\end{problem}
\begin{adjustwidth}{0.7cm}{}
\color{blue}
\begin{proof}[Solution]
Suppose that $X=(X_{i,j})_{i,j=1,\cdots,n}$ and $Y=(Y_{i,j})_{i,j=1,\cdots,n}$, then
\begin{align*}
\mathrm{Trace}(XY)&=\mathrm{Trace}\left(\left(\sum_{k=1}^{n}X_{i,k}Y_{k,j}\right)_{i,j=1,\cdots,n}\right)\\
&=\sum_{i=1}^{n}\sum_{k=1}^{n}X_{i,k}Y_{k,i}
\end{align*}
and similarly
\begin{align*}
\mathrm{Trace}(YX)&=\mathrm{Trace}\left(\left(\sum_{k=1}^{n}Y_{i,k}X_{k,j}\right)_{i,j=1,\cdots,n}\right)\\
&=\sum_{i=1}^{n}\sum_{k=1}^{n}Y_{i,k}X_{k,i}.
\end{align*}
\color{black}
\end{proof}
\end{adjustwidth}

\end{document} 