\documentclass{article}
\usepackage{amsmath}
\usepackage{amsfonts}
\usepackage{amssymb}
\usepackage{amsmath}
\usepackage{amsthm}
\usepackage{bm}
\usepackage{changepage}
\usepackage{esint}
\usepackage{framed}
\usepackage{geometry}
\usepackage{inputenc}
\usepackage{mathrsfs}
\usepackage{mathtools}
\usepackage{marvosym}
\usepackage{times}
\usepackage{xcolor}

\title{Math 4500 HW \#03 Solutions}
\author{Instructor: Birgit Speh\\ TA: Guanyu Li}
\date{}
\geometry{left=2cm,right=2cm,top=2.5cm,bottom=2.5cm}

\theoremstyle{definition}
\newtheorem{problem}{Problem}
\theoremstyle{plain}
\newtheorem*{remark}{Remark}

\begin{document}

\maketitle\par

\emph{This solution set is not error-free. Please email me (gl479\MVAt cornell.edu) if you spot any errors or typos!}

\begin{problem}[Exercise 2.5.4 (5 pts)]By the reflection formula, the product
\begin{displaymath}
q\mapsto u_4\bar{u}_3u_2\bar{u}_1q\bar{u}_1u_2\bar{u}_3u_4
\end{displaymath}
is a reflection in the hyperplanes orthogonal to $u_1,u_2,u_3,u_4$ respectively. Check that $u_4\bar{u}_3u_2\bar{u}_1=i$ and $\bar{u}_1u_2\bar{u}_3u_4=1$, so the product of the four reflections is indeed $q\mapsto iq$.
\end{problem}
\begin{adjustwidth}{0.7cm}{}
\color{blue}
\begin{proof}[Solution]
Here we know that $u_1=i,u_2=\frac{-1+i}{\sqrt{2}},u_3=k,u_4=\frac{-j+k}{\sqrt{2}}$, hence $\bar{u}_1=-i,\bar{u}_2=\frac{-1-i}{\sqrt{2}},\bar{u}_3=-k,\bar{u}_4=\frac{j-k}{\sqrt{2}}$. Thus
\begin{displaymath}
u_4\bar{u}_3u_2\bar{u}_1=\frac{-j+k}{\sqrt{2}}(-k)\frac{-1+i}{\sqrt{2}}i=\frac{1+i}{\sqrt{2}}\frac{-1+i}{\sqrt{2}}i=i
\end{displaymath}
and
\begin{displaymath}
\bar{u}_1u_2\bar{u}_3u_4=-i\frac{-1+i}{\sqrt{2}}(-k)\frac{-j+k}{\sqrt{2}}=\frac{j+k}{\sqrt{2}}\frac{-j+k}{\sqrt{2}}=1.
\end{displaymath}
\color{black}
\end{proof}
\end{adjustwidth}

\begin{problem}[Exercise 2.7.1 (5 pts)]Check that $q\mapsto u^{-1}qu$ is an automorphism of $\mathbb{H}$ for any unit quaternion $u$.
\end{problem}
\begin{adjustwidth}{0.7cm}{}
\color{blue}
\begin{proof}[Solution]It suffices to check that $q\mapsto u^{-1}qu$ is an isomorphism from $\mathbb{H}$ to $\mathbb{H}$.\par
For any two quaternion $q_1,q_2$, we have
\begin{displaymath}
u^{-1}(q_1q_2)u=u^{-1}(q_1(uu^{-1})q_2)u=(u^{-1}q_1u)(u^{-1}q_2u)
\end{displaymath}
and
\begin{displaymath}
u^{-1}(q_1+q_2)u=(u^{-1}q_1+u^{-1}q_2)u=u^{-1}q_1u+u^{-1}q_2u
\end{displaymath}
since the multiplication is a homomorphism. These imply that $q\mapsto u^{-1}qu$ is a homomorphism $\mathbb{H}\to\mathbb{H}$.\par
If
\begin{displaymath}
u^{-1}q_1u=u^{-1}q_2u
\end{displaymath}
for two quaternion $q_1,q_2$, then by multiplying $u$ and $u^{-1}$, we know $q_1=q_2$ and the map is injective. For any $q\in\mathbb{H}$, the map sends $uqu^{-1}$ to $q$. Hence it is surjective$.^{[5]}$
The continuity is derived from the fact that matrix multiplication is continuous.
\color{black}
\end{proof}
\end{adjustwidth}

\begin{problem}[Exercise 3.1.2 (5 pts)]Give an example of a matrix in $O(3)$ that is not in $SO(3)$, and interpret it geometrically.
\end{problem}
\begin{adjustwidth}{0.7cm}{}
\color{blue}
\begin{proof}[Solution]Consider
\begin{displaymath}
A=\begin{pmatrix}
1&&\\ &1&\\ &&-1
\end{pmatrix}
\end{displaymath}
is an element in $O(n)$. But $\det(A)=-1$, which means $A\notin SO(3)$. Geometrically it is the reflection about the $xOy$ plane.
\color{black}
\end{proof}
\end{adjustwidth}

\begin{problem}[Exercise 3.2.1 (10 pts)]Bearing in mind that matrix multiplication is a continuous operation, show that if there are continuous paths in $G$ from $I$ to $A\in G$ and to $B\in G$ then there is a continuous path in $G$ from $A$ to $AB$.
\end{problem}
\begin{adjustwidth}{0.7cm}{}
\color{blue}
\begin{proof}[Solution]Since $I$ and $B$ are path-connected, we have a path
\begin{displaymath}
\alpha:[0,1]\to G
\end{displaymath}
s.t. $\alpha$ is continuous and $\alpha(0)=I$, $\alpha(1)=B$. Since the multiplication of matrix is continuous, we have another path
\begin{align*}
\bar{\alpha}:[0,1]&\to G\\
t&\mapsto A\cdot\alpha(t).
\end{align*}
Since $\bar\alpha(0)=A$ and $\bar\alpha(1)=AB$, we know $\bar{\alpha}$ is a path connecting $A$ and $AB$. Hence $A$ and $AB$ are path-connected.
\color{black}
\end{proof}
\end{adjustwidth}
\begin{remark}
Here it is not necessary to know the path-connectedness of $I$ and $B$. However, the textbook did not give us the definition of path-connectedness, which made a lot of people confused. The formal definition is in a space $X$, two points $x,y$ are said to be path-connected if there is a continuous map
\begin{displaymath}
\alpha:[0,1]\to X
\end{displaymath}
s.t. $\alpha(0)=x$ and $\alpha(1)=y$.
\end{remark}

\begin{problem}[Exercise 3.2.2 (5 pts)]Similarly show that if  there is a continuous path in $G$ from $I$ to $C$, then there is also a continuous path from $C^{-1}$ to $I$.
\end{problem}
\begin{adjustwidth}{0.7cm}{}
\color{blue}
\begin{proof}[Solution]By previous exercise, put $B=C$ and $A=C^{-1}$.
\color{black}
\end{proof}
\end{adjustwidth}

\begin{problem}[Exercise 2.7.2 (0 pts)]Prove that an automorphism $\rho$ of $\mathbb{H}$ preserves 0 and differences.
\end{problem}
\begin{adjustwidth}{0.7cm}{}
\color{blue}
\begin{proof}[Solution]Consider
\begin{displaymath}
\rho(0)=\rho(0+0)=\rho(0)+\rho(0),
\end{displaymath}
thus $\rho(0)=0$. For any two $p,q\in\mathbb{H}$, since $\rho$ is an automorphism, $\rho(p)=\rho(p-q+q)=\rho(p-q)+\rho(q)$, thus it preserves the differences.
\color{black}
\end{proof}
\end{adjustwidth}

\begin{problem}[Exercise 2.7.3 (0 pts)]Prove that an automorphism $\rho$ of $\mathbb{H}$ preserves 1 and quotients.
\end{problem}
\begin{adjustwidth}{0.7cm}{}
\color{blue}
\begin{proof}[Solution]Similar to 2.7.2.
\color{black}
\end{proof}
\end{adjustwidth}

\begin{problem}[Exercise 2.7.4 (0 pts)]Prove that an automorphism $\rho$ of $\mathbb{H}$ is a $\mathbb{R}$ linear map.
\end{problem}
\begin{adjustwidth}{0.7cm}{}
\color{blue}
\begin{proof}[Solution]First for any $m,n\in\mathbb{Z}$ s.t. $n\neq0$, we have
\begin{displaymath}
\rho(mq)=\rho(q+\cdots+q)=\rho(q)+\cdots+\rho(q)=m\rho(q)
\end{displaymath}
and
\begin{displaymath}
n\rho\left(\frac{m}{n}q\right)=\rho\left(n\frac{m}{n}q\right)=\rho\left(mq\right)=m\rho(q),
\end{displaymath}
therefore $\rho\left(\frac{m}{n}q\right)=\frac{m}{n}\rho(q)$.\par
For any real number $r$, there exists a sequence of rational numbers $\{r_k\}_{k\in\mathbb{N}}$ s.t. $r_k\to r$ as $k\to\infty$. Hence by the continuity of $\rho$,
\begin{displaymath}
\rho(rq)=\rho\left(\left(\lim_{k\to+\infty}r_k\right)q\right)=\lim_{k\to+\infty}\rho(r_kq)=\lim_{k\to+\infty}r_k\rho(q)=r\rho(q).
\end{displaymath}
\color{black}
\end{proof}
\end{adjustwidth}

\end{document} 