\documentclass{article}
\usepackage{amsmath}
\usepackage{amsfonts}
\usepackage{amssymb}
\usepackage{amsmath}
\usepackage{amsthm}
\usepackage{bm}
\usepackage{changepage}
\usepackage{esint}
\usepackage{framed}
\usepackage{geometry}
\usepackage{inputenc}
\usepackage{mathrsfs}
\usepackage{mathtools}
\usepackage{marvosym}
\usepackage{times}
\usepackage{xcolor}

\title{Math 4500 HW \#10 Solutions}
\author{Instructor: Birgit Speh\\ TA: Guanyu Li}
\date{}
\geometry{left=2cm,right=2cm,top=2.5cm,bottom=2.5cm}

\theoremstyle{definition}
\newtheorem{problem}{Problem}
\theoremstyle{plain}
\newtheorem*{remark}{Remark}

\begin{document}

\maketitle\par

\emph{This solution set is not error-free. Please email me (gl479\MVAt cornell.edu) if you spot any errors or typos!}

\begin{problem}[Exercise 8.2.2 (7 pts)]
Show that $GL_n(\mathbb{C})$ is an open subset of $M_n(\mathbb{C})$.
\end{problem}
\begin{adjustwidth}{0.7cm}{}
\color{blue}
\begin{proof}[Solution]
Considering the continuous function
\begin{displaymath}
\det:M_n(\mathbb{C})\to\mathbb{C},
\end{displaymath}
$GL_n(\mathbb{C})$ is the preimage of $\mathbb{C}-\{0\}$. The continuity tells us that $GL_n(\mathbb{C})$ is open.
\color{black}
\end{proof}
\end{adjustwidth}

\begin{problem}[Exercise 8.3.4 (7 pts)]
Give an example of a continuous function $f$ on $\mathbb{R}$ and a set $S$ s.t.
\begin{displaymath}
f(\bar{S})\neq\overline{f(S)},
\end{displaymath}
where $\bar{S}$ is the closure of $S$.
\end{problem}
\begin{adjustwidth}{0.7cm}{}
\color{blue}
\begin{proof}[Solution]
Consider
\begin{displaymath}
f(x)=\arctan x,
\end{displaymath}
and $S=\mathbb{R}$, then $(-\frac{\pi}{2},\frac{\pi}{2})=f(S)=f(\bar{S})\subsetneqq[-\frac{\pi}{2},\frac{\pi}{2}]=\overline{f(S)}$.
\color{black}
\end{proof}
\end{adjustwidth}

\begin{problem}[Exercise 8.4.5 (10 pts)]
Show that $GL(n,\mathbb{C})$ and $SL(n,\mathbb{C})$ are not compact.
\end{problem}
\begin{adjustwidth}{0.7cm}{}
\color{blue}
\begin{proof}[Solution]
First it is not hard to prove that a continuous image of a compact set is still compact (you could prove it yourself). Hence we consider the continuous map $\det:GL(n,\mathbb{C})\to\mathbb{C}$, its image is $\mathbb{C}-\{0\}$ which is not compact, hence $GL(n,\mathbb{C})$ is not compact.\par
For the second part, we shall use the fact that a subset $C$ is compact in an Euclidean space if and only if it is bounded and closed. But the matrices
\begin{displaymath}
A_k:=\begin{pmatrix}k&&&&\\ &\frac{1}{k}&&&\\ &&1&&\\ &&&\ddots&\\ &&&&1
\end{pmatrix}
\end{displaymath}
are element in $SL_n(\mathbb{C})$ however $\lVert A_k\rVert>k\to+\infty$ as $k\to+\infty$. Hence $SL_n(\mathbb{C})$ is not bounded.
\color{black}
\end{proof}
\end{adjustwidth}

\begin{problem}[Exercise 8.6.1 (6 pts)]
Write $\begin{pmatrix}-1&1\\ &-1\end{pmatrix}$ as the product of two matrices in $SL(2,\mathbb{C})$ with entries $0,i$ or $-i$.
\end{problem}
\begin{adjustwidth}{0.7cm}{}
\color{blue}
\begin{proof}[Solution]
\begin{displaymath}
\begin{pmatrix}-1&1\\ &-1\end{pmatrix}=\begin{pmatrix}i&i\\ &-i\end{pmatrix}\begin{pmatrix}i&\\ &-i\end{pmatrix}.
\end{displaymath}
Notice that two matrices are traceless.
\color{black}
\end{proof}
\end{adjustwidth}

\begin{problem}[Exercise 8.6.2 (10 pts)]
Deduce from Exercise 8.6.1 and Exercise 5.6.4 that $\begin{pmatrix}-1&1\\ &-1\end{pmatrix}=e^{X_1}e^{X_2}$ for some $X_1,X_2\in T_I(SL(2,\mathbb{C}))$.
\end{problem}
\begin{adjustwidth}{0.7cm}{}
\color{blue}
\begin{proof}[Solution]
Suppose that $\begin{pmatrix}i&i\\ &-i\end{pmatrix}=e^{X_1}$. By Caylay-Hamilton, we can use the result from Problem 5.6.4.,
\begin{displaymath}
e^{X_1}=\cos(\sqrt{\det(X_1)})I+\frac{\sin(\sqrt{\det(X_1)})}{\sqrt{\det(X_1)}}X_1.
\end{displaymath}
This identity is actually gives us a system of linear equations. Thus we have that
\begin{displaymath}
X_1=\frac{\pi}{2}\begin{pmatrix}i&i\\ &-i\end{pmatrix}.
\end{displaymath}
Similarly we have $X_2=\frac{\pi}{2}\begin{pmatrix}i&\\ &-i\end{pmatrix}$.
\color{black}
\end{proof}
\end{adjustwidth}

\begin{problem}[Exercise 8.1.2 (0 pts)]
\end{problem}
\begin{adjustwidth}{0.7cm}{}
\color{blue}
\begin{proof}[Solution]
\color{black}
\end{proof}
\end{adjustwidth}

\begin{problem}[Exercise 8.3.1 (0 pts)]
\end{problem}
\begin{adjustwidth}{0.7cm}{}
\color{blue}
\begin{proof}[Solution]
\color{black}
\end{proof}
\end{adjustwidth}

\begin{problem}[Exercise 8.3.2 (0 pts)]
\end{problem}
\begin{adjustwidth}{0.7cm}{}
\color{blue}
\begin{proof}[Solution]
\color{black}
\end{proof}
\end{adjustwidth}

\begin{problem}[Exercise 8.5.1 (0 pts)]
\end{problem}
\begin{adjustwidth}{0.7cm}{}
\color{blue}
\begin{proof}[Solution]
\color{black}
\end{proof}
\end{adjustwidth}

\begin{problem}[Exercise 8.5.2 (0 pts)]
\end{problem}
\begin{adjustwidth}{0.7cm}{}
\color{blue}
\begin{proof}[Solution]
\color{black}
\end{proof}
\end{adjustwidth}

\end{document} 